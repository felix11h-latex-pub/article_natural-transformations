\documentclass[a4paper]{amsart}

%-------packages-------------------------%
\usepackage{amsmath}
\usepackage{xypic}  				%for strange reason I need this to make the two cell diagrams				
\usepackage[2cell]{xy} 			%for commutative diagrams%

\usepackage{color,hyperref}
\definecolor{darkblue}{rgb}{0.0,0.0,0.3}                      %pdf book marks the way I like%
\hypersetup{pdftex=true, colorlinks=true, breaklinks=true, linkcolor=darkblue, menucolor=darkblue, pagecolor=darkblue, urlcolor=darkblue}
%-----------style------------------------%

\addtolength{\parskip}{\baselineskip} %Abs�tze im Text werden auch tats�chlich zu Abs�tzen%


%----------new--commands-----------------%

\newcommand{\C}{\mathcal{C}}
\newcommand{\D}{\mathcal{D}}
\newcommand{\F}{\mathcal{F}}
\newcommand{\G}{\mathcal{G}}
\newcommand{\ve}{\varepsilon}

% the following code will uplift the \maketitle title. In standard it is way too low.
\makeatletter % wegen @ in den Befehsnamen
\renewcommand*\@maketitle{%
  \normalfont\normalsize
  \@adminfootnotes
  \@mkboth{\@nx\shortauthors}{\@nx\shorttitle}%
% (SCHW) auskommentiert:  \global\topskip42\p@\relax % 5.5pc   "   "   "     "     "
  \@settitle
  \ifx\@empty\authors \else \@setauthors \fi
  \ifx\@empty\@dedicatory
  \else
    \baselineskip18\p@
    \vtop{\centering{\footnotesize\itshape\@dedicatory\@@par}%
      \global\dimen@i\prevdepth}\prevdepth\dimen@i
  \fi
  \@setabstract
  \normalsize
  \if@titlepage
    \newpage
  \else
    \dimen@34\p@ \advance\dimen@-\baselineskip
    \vskip\dimen@\relax
  \fi
} % end \@maketitle
\makeatother





\begin{document}

%------------header----------------------%

\title{Vertical and horizontal composition}
%\author{Felix Hoffmann}% %for now I decided against it%

\begin{abstract}
There are two ways to connect natural transformations between functors. One is called vertical composition and the other is called horizontal composition. Both are explained in this article.
\end{abstract}

\maketitle
\pagestyle{empty} %no pagenumbers and titles%
\thispagestyle{empty}

%-------end header-----------------------%


Given natural transformations $\eta : F \to G$, $\ve: G \to H$ 
%
%
\Large
%
\begin{equation*}
%
\UseAllTwocells
\xymatrix @+=3cm{\C \ruppertwocell<9>^{F}{<-2.5>_{\mbox{ } \eta}} \ar[r]|G \rlowertwocell<-9>_{H}{<2.5>^{\mbox{ }\ve}} & \D}
%
\end{equation*}
%
\normalsize
%
%
%
we obtain a natural transformation $\ve \eta : F \to H$ with components $(\ve \eta)_x := \ve_x \circ \eta_x$.

Then $\ve \eta$ is called \emph{vertical composition} of $\eta$ and $\ve$. To see that $\ve \eta$ is indeed a natural transforamation one only needs to consult the following diagram:

\bigskip

\begin{minipage}{0.45\textwidth}
\begin{equation*}
\xymatrix @+=1.4cm{F(X) \ar[r]^{F(f)} \ar[d]_{\eta_X} &F(Y)\ar[d]_{\eta_Y} \\
                   G(X) \ar[r]^{G(f)}\ar[d]_{\ve_X} &G(Y) \ar[d]_{\ve_Y}  \\
                   H(X) \ar[r]^{H(f)}               &H(Y)}
\end{equation*}
\end{minipage}%
\begin{minipage}{0.45\textwidth}
\begin{align*}
(\ve \eta)_Y \circ F(f) &= \ve_Y \circ \eta_Y \circ F(f) \\
                        &= \ve_Y \circ G(f) \circ \eta_X \\
                        &= H(f) \circ \ve_X \circ \eta_X \\
                        &= H(f) \circ (\ve \eta)_X
\end{align*}
\end{minipage}

\bigskip
\bigskip

Given natural transformations

\begin{equation*}
\UseAllTwocells
\xymatrix @+=3cm{\C \ruppertwocell<4.5>^{F}{<0>_{\eta}} \rlowertwocell<-4.5>_{G}{\omit} & \D \ruppertwocell<4.5>^{J}{<0>_\ve}	\rlowertwocell<-4.5>_{K}{\omit} & \mathcal{E}}
\end{equation*}
\bigskip

$\eta: F \to G$, $\ve: J \to K$ we obtain a natural transformation $\ve \eta: JF \to KG$ by \emph{horizontal composition}:

The components of $\ve \eta$ are given by 
\begin{equation*}
(\ve \eta)_X = \ve_{G(X)} \circ  J(\eta_X)
\end{equation*}

\small
\begin{minipage}{0.45\textwidth}
\begin{equation*}
\xymatrix @+=1.4cm{JF(X) \ar[r]^{JF(f)} \ar[d]_{J(\eta_X)}  & JF(Y) \ar[d]^{J(\eta_Y)} \\
								 	 JG(X) \ar[r]^{JG(f)} \ar[d]_{\ve_{G(X)}} & JG(Y) \ar[d]^{\ve_{G(Y)}}\\
					         KG(X) \ar[r]^{KG(f)}										 &KG(Y)												}
\end{equation*}
\end{minipage}%
\begin{minipage}{0.45\textwidth}

\begin{align*}
(\ve \eta)_Y \circ JF(f) &= \ve_{G(Y)} \circ J(\eta_Y) \circ JF(f)\\
												 &= \ve_{G(Y)} \circ J(\eta_Y \circ F(f) 		\\
												 &= \ve_{G(Y)} \circ JG(f) \circ J(\eta_X)  \\
												 &= KG(f) \circ \ve_{G(X)} J(\eta_X)				\\
												 &= KG(f) \circ (\ve \eta)_X \end{align*}

\normalsize												 
\end{minipage}

\bigskip

\bibliographystyle{amsalpha}
\begin{thebibliography}{00}
\bibitem[PlanMath]{PlanMath} \href{http://planetmath.org/?method=js&from=objects&id=11112&op=getobj}{PlanetMath.org - compositions of natural transformations}  
\end{thebibliography}


\end{document}


